\documentclass[11pt]{article}
\input{.tex_default_env.tex}

\usepackage[backend=biber, autopunct=true, authordate]{biblatex-chicago}  

\title{Liquidity Estimation}
\date{\today}
\author{Paul Sangrey and Christos Makridis}

\begin{document}

\maketitle 


\section{Asset Pricing Problem}


\begin{gather}
    V\left(w_t\right) = \max u(c_t) + \beta \E_{market}[ V\left(w_{t+1}\right)] \\
    \text{s.t.}\ w_{t+1} = (1 + r^r_{t+1}) (w_t - c_t - s_t) + (1 + r^s_{t+1}) s_t  \\
    \text{s.t.}\  s_t \geq \text{margin\_constraint}\ \cdot w_t  \label{eqn:liquidity_constraint}
\end{gather}

The effective price of liquidity is the Lagrange multiplier for \cref{eqn:liquidity_constraint}. 
The above equation implies that we have an Euler equation of the following form.

\begin{equation}
    \E\left[ \frac{u'\left(c_{t+1}\right)}{u'\left(c_t\right)} \cdot \frac{\dif P^M}{\dif P^{True}} \cdot 
    \text{\textit{liquidity\_term}} \cdot R_{t+1} \right] = 1 
\end{equation}


\section{Identification Strategy}



\begin{itemize}
    \item Put the above expression logs and in continuous time, and difference.
    \item Use your consumption data to control for the change in the ratio of marginal utilities.
    \item Use volatility to control for the change in beliefs. (We should be able to get the likelihood's in
        closed form.)
    \item We observe returns. 
    \item That leaves us with the liquidity term. (Or more generally a constraint term.)
    \item This probably isn't enough to truly identify the model because there's probably a drift in the ratio of
        probability term, but we can probably produce rather tight bounds.
\end{itemize}

\section{\qt{Division of Labor}}

\begin{itemize}
    \item I know the high-frequency finance / information theory relevant to identify the model.
    \item You're know a bunch about regression-style identification.
    \item It might be a good idea to get a straight-up finance person on the project.
    \item We can all learn from each other.
\end{itemize}




\section{Regressions}


\begin{equation}
    \frac{R_t}{\sigma_t} = \beta_0 + \beta_1 \Delta c_t + \beta_3 \Delta c_t^2 + \beta_2 \sigma^2_t + \beta_3
    \Delta Beliefs_t  + \beta_4 \Delta Beliefs^2_t + \eta_t
\end{equation}

The innovation to the regression above $\eta_t$ is a proxy for liquidity changes.
So we want to sum it, and compare it to various other liquidity measures.

The $\sigma_t$ under the $R_T$ is a heteroskedasticity adjustment.
In theory $R_t$ should be the difference between the stock market and the risk-free rate, but because of the
zero-lower bound, the risk-free rate is essentially zero. 
The square terms are in there because of Ito's Lemma. 
The $\sigma_t^2$ is acting as a proxy for the rate of change in beliefs.



\end{document}
